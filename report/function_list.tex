\documentclass[10pt,a4paper]{article}
\usepackage[utf8]{inputenc}
\usepackage[turkish]{babel}
\usepackage[left=0.8in, right=1.0in, top=0.8in, bottom=.8in]{geometry}
\usepackage{amsmath}
\usepackage{amsfonts}
\usepackage{amssymb}
\usepackage{graphicx}
\usepackage{tikz}
\usepackage{fancyhdr}
\pagestyle{fancy}

\fancyhf{}
\fancyhead[L]{Robotes Robotik Çözümler}
\fancyhead[c]{Tren Dinamik Model Fonksiyon Listesi}
\rhead{ \fancyplain{}{\today} }
\rfoot{ \fancyplain{}{\thepage} }


\author{Robotes Robotik Çözümler}
\title{Tren Dinamik Model Fonksiyon Listesi}

\begin{document}

\maketitle
\newpage

\tableofcontents
\newpage

\section{Çözücü}

\subsection{Diferansiyel Denklem Çözücü}

\begin{description}
\item[Tanım ve Amaç:] 
\item[Tagler:]
\item[Fonksiyon:]
\item[Testler:]
\item[Kabuller:]
\end{description}

\newpage

\section{Lokomotif Motor}

\subsection{Motor Kuvvet Hesabı}
\begin{description}
\item[Tanım ve Amaç:] 
\item[Tagler:]
\item[Fonksiyon:]
\item[Testler:]
\item[Kabuller:]
\end{description}
\newpage

\subsection{Dinamik Fren Hesabı}
\begin{description}
\item[Tanım ve Amaç:] 
\item[Tagler:]
\item[Fonksiyon:]
\item[Testler:]
\item[Kabuller:]
\end{description}
\newpage

\subsection{Güç Hesabı}
\begin{description}
\item[Tanım ve Amaç:] 
\item[Tagler:]
\item[Fonksiyon:]
\item[Testler:]
\item[Kabuller:]
\end{description}
\newpage

\subsection{Verim}
\begin{description}
\item[Tanım ve Amaç:] 
\item[Tagler:]
\item[Fonksiyon:]
\item[Testler:]
\item[Kabuller:]
\end{description}
\newpage

\newpage
\section{Kayıp Kuvvetler}
Bu bölümde tren üzerinde oluşan kayıp kuvvetler incelenmektedir. Ayrıca trenin kayma ve patinaj durumları anlatılmıştır.

\subsection{Seyir Direnci}
\begin{description}

\item[Tanım ve Amaç:] Seyir direnci bir aracın üzene etkiyen rüzgar, tekerlek sürtünmeleri vb kayıp kuvvetleri hesaplarç Seyir direnci araç çeşidine göre değişmektedir. Bu değişim araçların yapılarından ve konumlarından kaynaklanmaktadır. 

\item[Tagler:] sürtünme, seyir direnci, kayıp kuvvet

\item[Çalışma Şekli:] Lokomotif, yolcu vagonu ve yük vagonu için ayrı hesaplar vardır. Bu hesaplar !!!!! kitabından alınmıştır. 

\begin{itemize}
\item \textbf{Lokomotif Seyir Direnci}
\begin{equation}
F_{seyir} = (6.5 \times m_t) + (130 \times n_a) + (0.1 \times m_t \times \dot{x}_{km}) + (0.3 \times \dot{x}^2_{km})
\end{equation}

\item \textbf{Yük Vagonu Seyir Direnci}
\begin{equation}
F_{seyir} = (2 + \dfrac{0.057 \times \dot{x}^2_{km} }{100})\times m_t \times 10
\end{equation}

\item \textbf{Yolcu Vagonu Seyir Direnci}
\begin{equation}
F_{seyir} = (1.986 + 0.00932 \times \dot{x}_km + 0.000161 \dot{x}^2_{km}) \times m_t
\end{equation}
\end{itemize}

\item[Ayarlar ve Değişkenler:] Değişkenler aracın modeline göre değişmektedir. Ayarlanması gereken bir değer yoktur.
\begin{description}
\item[$n_a$:]araç üzerindeki aks sayısı, teker çifti (birimi yok) [number of axels]
\item[$m_t$:]aracın kütlesi (ağırlığı????) (birimi????)
\item[$\dot{x}_k$:] aracın hızı (km/sn)
\end{description}

\item[İlgili Fonksiyonlar:] Lokomotif ve araçlar için farklı olduğundan sınıflar içinde hesaplanmakta, denklem değişimlerinin kolay yapılması için hesaplar farklı bir sınıfta yapılmıştır.
\begin{itemize}
\item RSTrainVehicle sınıfında getRollingResistanceForce ve setRollingResistanceForce (fonksiyonlara referans ver) fonksiyonları ile get set işlemleri gerçekleniyor. 
\item RSTrainLocomotiveRollingResistanceCalculator sınıfı ve RSTrainCarRollingResistanceCalculator sınıfları ile seyir direnci hesaplanıyor.
\end{itemize}

\item[Testler:]Değerlerin doğru hesaplandığında dair bir test yapılmıştır.

\item[Kabuller:]Denklemler empirik yöntemlerle bulunmuştur. Kuvvet değerleri ufak olduğundan sistem üzerinde büyük etkileri yoktur. 

\end{description}
\newpage

\subsection{Kurb Direnci}
\begin{description}
\item[Tanım ve Amaç:] Kurb direnci kurblarda tren üzerine etki eden bir kuvvettir. Bu kuvvet kurbun yarıçap uzunluğuna bağlıdır.
\item[Tagler:]sürtünme, kurb direnci, kayıp kuvvet
\item[Çalışma Şekli:] Denklem !!!! kitabından alınmıştır.

\begin{equation}
F_{kurb} = \dfrac{6500}{r_c - 55} \times m_t
\end{equation}

\item[Ayarlar ve Değişkenler:] Ayarlar her araç için aynıdır. Yolun dönüş yarıçapı ve kütleye bağlıdır.
\begin{description}
\item[$r_c$:] Kurb yarıçapı (km)
\item[$m_t$:] Aracın kütlesi, ağırlığı
\end{description}


\item[Testler:] Değerlerin doğru hesaplandığına dair bir test yapılmıştır.
\item[Kabuller:] Denklemler empirik yöntemlerle bulunmuştur. Kuvvet değerleri ufak olduğu için sistem üzerinde büyük etkileri yoktur.
\end{description}
\newpage

\subsection{Rampa Direnci}
\begin{description}
\item[Tanım ve Amaç:] 
\item[Tagler:]
\item[Fonksiyon:]
\item[Testler:]
\item[Kabuller:]
\end{description}
\newpage

\subsection{Kayma}
\begin{description}
\item[Tanım ve Amaç:] 
\item[Tagler:]
\item[Fonksiyon:]
\item[Testler:]
\item[Kabuller:]
\end{description}

\newpage

\section{Yol}

\subsection{Yol Durumunun Hesaplanması}
\begin{description}
\item[Tanım ve Amaç:] 
\item[Tagler:]
\item[Fonksiyon:]
\item[Testler:]
\item[Kabuller:]
\end{description}
\newpage

\subsection{Yol Bilgisinin Alınması}
\begin{description}
\item[Tanım ve Amaç:] 
\item[Tagler:]
\item[Fonksiyon:]
\item[Testler:]
\item[Kabuller:]
\end{description}

\newpage

\section{Tren Oluşturma}
\begin{description}
\item[Tanım ve Amaç:] 
\item[Tagler:]
\item[Fonksiyon:]
\item[Testler:]
\item[Kabuller:]
\end{description}
\newpage

\subsection{Lokomotif ve Vagon Ekleme}
\begin{description}
\item[Tanım ve Amaç:] 
\item[Tagler:]
\item[Fonksiyon:]
\item[Testler:]
\item[Kabuller:]
\end{description}
\newpage

\subsection{Araçları Bağlama}
\begin{description}
\item[Tanım ve Amaç:] 
\item[Tagler:]
\item[Fonksiyon:]
\item[Testler:]
\item[Kabuller:]
\end{description}
\newpage

\subsection{Akuple}
\begin{description}
\item[Tanım ve Amaç:] 
\item[Tagler:]
\item[Fonksiyon:]
\item[Testler:]
\item[Kabuller:]
\end{description}
\newpage

\subsection{Ranfor}
\begin{description}
\item[Tanım ve Amaç:] 
\item[Tagler:]
\item[Fonksiyon:]
\item[Testler:]
\item[Kabuller:]
\end{description}

\newpage

\section{Hava Freni}

\subsection{Kontrol Valfı}
\begin{description}
\item[Tanım ve Amaç:] 
\item[Tagler:]
\item[Fonksiyon:]
\item[Testler:]
\item[Kabuller:]
\end{description}
\newpage

\subsection{Yedek Hava Deposu}
\begin{description}
\item[Tanım ve Amaç:] 
\item[Tagler:]
\item[Fonksiyon:]
\item[Testler:]
\item[Kabuller:]
\end{description}
\newpage

\subsection{Fren Silindiri}
\begin{description}
\item[Tanım ve Amaç:] 
\item[Tagler:]
\item[Fonksiyon:]
\item[Testler:]
\item[Kabuller:]
\end{description}
\newpage

\subsection{Fren Kuvveti}
\begin{description}
\item[Tanım ve Amaç:] 
\item[Tagler:]
\item[Fonksiyon:]
\item[Testler:]
\item[Kabuller:]
\end{description}
\newpage

\subsection{Fren Hataları}
\begin{description}
\item[Tanım ve Amaç:] 
\item[Tagler:]
\item[Fonksiyon:]
\item[Testler:]
\item[Kabuller:]
\end{description}
\newpage

\subsection{Modrabl}
\begin{description}
\item[Tanım ve Amaç:] 
\item[Tagler:]
\item[Fonksiyon:]
\item[Testler:]
\item[Kabuller:]
\end{description}
\newpage

\subsection{Purjör}
\begin{description}
\item[Tanım ve Amaç:] 
\item[Tagler:]
\item[Fonksiyon:]
\item[Testler:]
\item[Kabuller:]
\end{description}
\newpage

\subsection{Makinist Musluğu}
\begin{description}
\item[Tanım ve Amaç:] 
\item[Tagler:]
\item[Fonksiyon:]
\item[Testler:]
\item[Kabuller:]
\end{description}
\newpage

\subsection{Kondüvit Hattı}
\begin{description}
\item[Tanım ve Amaç:] 
\item[Tagler:]
\item[Fonksiyon:]
\item[Testler:]
\item[Kabuller:]
\end{description}
\newpage

\subsection{Ana Hava Deposu ve Kompresör}
\begin{description}
\item[Tanım ve Amaç:] 
\item[Tagler:]
\item[Fonksiyon:]
\item[Testler:]
\item[Kabuller:]
\end{description}

\newpage

\end{document}