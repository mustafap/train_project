\documentclass[10pt,a4paper]{article}
\usepackage[utf8]{inputenc}
\usepackage[turkish]{babel}
\usepackage[left=0.8in, right=1.0in, top=0.8in, bottom=.8in]{geometry}
\usepackage{amsmath}
\usepackage{amsfonts}
\usepackage{amssymb}
\usepackage{graphicx}
\usepackage{tikz}
\usepackage{fancyhdr}
\pagestyle{fancy}

\fancyhf{}
\fancyhead[L]{Robotes Robotik Çözümler}
\fancyhead[c]{Tren Dinamik Model Fonksiyon Listesi}
\rhead{ \fancyplain{}{\today} }
\rfoot{ \fancyplain{}{\thepage} }


\author{Robotes Robotik Çözümler}
\title{Tren Dinamik Model Fonksiyon Listesi}

\begin{document}

\maketitle
\newpage

\tableofcontents
\newpage

\section{Çözücü}

\subsection{Diferansiyel Denklem Çözücü}


\newpage
\section{Lokomotif Motor}

\subsection{Motor Kuvvet Hesabı}

\subsection{Dinamik Fren Hesabı}

\subsection{Güç Hesabı}

\subsection{Verim}

\newpage
\section{Kayıp Kuvvetler}


\newpage
\section{Yol}

\subsection{Yol Durumunun Hesaplanması}

\subsection{Yol Bilgisinin Alınması}
\newpage

\section{Tren Oluşturma}

\subsection{Lokomotif ve Vagon Ekleme}

\subsection{Araçları Bağlama}

\subsection{Akuple}

\subsection{Ranfor}

\newpage
\section{Hava Freni}

\subsection{Kontrol Valfı}

\subsection{Yedek Hava Deposu}

\subsection{Fren Silindiri}

\subsection{Fren Kuvveti}

\subsection{Fren Hataları}

\subsection{Modrabl}

\subsection{Purjör}

\subsection{Makinist Musluğu}

\subsection{Kondüvit Hattı}

\subsection{Ana Hava Deposu ve Kompresör}
\newpage

\end{document}